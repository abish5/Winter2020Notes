\documentclass[a4paper,12pt]{article}

\usepackage{tikz}
\usepackage[utf8]{inputenc}
\usepackage[english]{babel}
\usepackage{amssymb,amsmath,amsthm}

\newtheorem{theorem}{Theorem}
\newtheorem{example}{Example}
\newtheorem{problem}{Problem}
\newtheorem{lemma}{Lemma}

\usepackage{mathtools} % Bonus
\DeclarePairedDelimiter\norm\lVert\rVert

\title{Feb 25 Lecture}
\author{COMP2804 Winter 2020}

\begin{document}

\section{Independent sets}
$\textbf{Def}$\\
    Two events A and B are independent if $Pr(A \cap B) = Pr(A) \cdot Pr(B)$.
$\textbf{Consquence}$\\
    If A and B are independent and $Pr(B) > 0$, then
\begin{align*}
    Pr(A | B) = \frac{Pr(A \cap B)}{Pr(B)} = \frac{Pr(A)\cdot Pr(B)}{Pr(B)}
\end{align*}
\begin{example}
    Rolling Dice $S = \{(D_{1}, D_{2}) : D_{1}, D_{2} \in \{1,2,3,4,5,6\}\}$\\
    $Pr(w) = \frac{1}{|S|} = \frac{1}{36}$\\ 
    A = $"D_{1} = 4" = \{(4,1),(4,2),(4,3),(4,4),(4,5),(4,6)\}$\\ 
    B = $"D_{2} = 3" = \{(1,3),(2,3),(3,3),(4,3),(5,3),(6,3)\}$\\ 
    $A \cap B = \{(4,3)\}$\\
    $Pr(A\cap B) = 1/36$\\
    $Pr(A) = |A|/36 = 6/36 = 1/6$\\
    $Pr(B) = |B|/36 = 6/36 = 1/6$\\
    $Pr(A) \cdot Pr(B) = 1/6 \cdot 1/6 = 1/36$\\
    Thus, A and B are independent.
\\\\
    Note: Do not use the word dependent when refering the sets that are not independent.
\\\\
    Now assume that $B = "D_{1} + D_{2} = 7" = \{(1,6),(2,5),(3,4),(4,3),(5,2),(6,1)\}$\\
    $A \cap B = \{(4,3)\}$\\
    $Pr(A\cap B) = 1/36$\\
    $Pr(A) = |A|/36 = 6/36 = 1/6$\\
    $Pr(B) = |B|/36 = 6/36 = 1/6$\\
    $Pr(A) \cdot Pr(B) = 1/6 \cdot 1/6 = 1/36$\\
    Thus, A and B are independent.\\
\end{example}
\begin{lemma}
    If A and B are independent. then A and $B^{\complement}$ are independent.
\end{lemma}
\begin{proof}[]
    Need to show 
    \begin{align*}
        Pr(A \cap B^{\complement}) &= Pr(A) \cdot Pr(B^\complement)\\
        &= Pr(A)(-Pr(B))
    \end{align*}
    \begin{align*}
        Pr(A) &= Pr(A \cap B) + Pr(A \cap B^\complement)\\
        &= Pr(A)\cdot Pr(B) + Pr(A \cap B^\complement)\\ 
        \leftrightarrow Pr(A \cap B^{\complement}) &= Pr(A) - Pr(A)\cdot Pr(B)\\
        &= Pr(A)(1-Pr(B))
    \end{align*}
\end{proof}


\end{document}